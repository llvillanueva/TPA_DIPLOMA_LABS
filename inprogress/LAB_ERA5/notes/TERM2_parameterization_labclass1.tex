
\section{Introduction}

\frame{
We are going to use an ECMWF output field in \alert{netcdf} format.

The contains variables on model levels for one day for: 
\begin{itemize}
\item $T$ : Temperature
\item $r$ : Relative Humidity (calculated as in the model)
\item $q$ : specific humidity
\item $clwc$ : Specific cloud liquid water content
\item $ciwc$ : Specific ice liquid water content
\item $cc$ : Cloud Cover
\end{itemize}

The last 3 cloud variables are produced by the ECMWF prognostic cloud
scheme described in \cite{tiedtke:93} and \cite{tompkins:07a}.
}

\frame{
The aim of this exercise is to code up some simple DIAGNOSTIC RH-based
cloud cover parameterizations and compare how they do relative to the
prognostic scheme.

We are going to do this using three tools:
\begin{itemize}
\item \texttt{ncview} : to quicklook view netcdf files 
\item \texttt{ncdump} : To examine netcdf file headers and contents
\item \texttt{cdo} : climate data operators, a programme to manipulate
  netcdf files (\url{https://code.zmaw.de/projects/cdo})
\end{itemize}
}
If you need to install any of these tools on a linux machine you can 


\section{Exercise 1: coding up the Sundqvist scheme}
\frame{
We will work together through exercise 1, while the subsequent
exercises you will perform in your groups.

\begin{enumerate}
\item code up the Sundqvist scheme (SS) to diagnose CC as a function of RH  
\item examine maps of the bias between ECMWF CC and Sundqvist CC
\item show a vertical profile of the RMS error (difference) between
  ECMWF and Sundqvist CC (a) globally (b) Tropics (lat $<20$) (c) NH
  extra tropics (lat$>30$ deg).
\end{enumerate}
}

\frame{
Reminder: Subgrid-scale fluctuations allow cloud
to form when $\overline{RH} < 1$.  $RH$ schemes formulize this by setting
a critical $RH$ (denoted $RH_{crit}$) at which cloud is assumed to form, and
then increase $C$ according to a monotonically increasing function of $RH$,
with $C$=1 identically when $RH$=1.  

One commonly used function was given by \cite{sundqvist:89}:
\begin{equation}
C=1-\sqrt{\frac{1-RH}{1-RH_{crit}}}
\label{sund_eqn}
\end{equation}
In this first exercise we will assume  $RH_{crit}$=0.6
}

\frame{
Get the datafile \texttt{ecmwf\_data.nc} here:

\texttt{\url{http://clima-dods.ictp.it/Users/tompkins/diploma/}}
\newline

Examine the file :  

\texttt{ncdump -h ecmwf\_data.nc}
\newline

View the file 

\texttt{ncview ecmwf\_data.nc}
\newline
}

\frame{
Using \texttt{cdo} - Download the manual from the clima-dods page, or
the \href{https://code.mpimet.mpg.de/projects/cdo/}{cdo page}. The general form of cdo is 

\texttt{cdo command input output}
\newline

e.g. Extract the fields we need:

\texttt{cdo selvar,r ecmwf\_data.nc rh.nc}

\texttt{cdo selvar,cc ecmwf\_data.nc cc.nc}
\newline

The RH field has units of percent - we change it to a fraction first:
\texttt{cdo divc,100 rh.nc  rhf.nc}

}

\frame{
Let's form the equation.  We can do this from the command line or by
putting the commands in a bash script.

How to calculate $1-RH$?

\texttt{cdo mulc,-1 rhf.nc rhm.nc}

\texttt{cdo addc,1 rhm.nc 1-rh.nc}
\newline

\texttt{cdo} also allows you to \alert{pipe} commands:

\texttt{cdo addc,1 -mulc,-1 rhf.nc 1-rh.nc}
\newline

Now we divide by $1-RH_{crit}$:

\texttt{cdo divc,0.4 1-rh.nc 1-rh\_d\_1-rhc.nc}
}

\frame{Now, remember that RH can exceed 1.0, or be less than
  $RH_{crit}$  and so we want to clip
  those negative values or values exceeding 1.  
To do this we need to define a mask - for negative values it is
straightforward:

\texttt{cdo gec,0 1-rh\_d\_1-rhc.nc mask.nc}

\texttt{cdo mul 1-rh\_d\_1-rhc.nc mask.nc temp1.nc}
\newline

for max values is a little more tricky.  This is one way, perhaps you
can find a quicker one:

\texttt{cdo gtc,1 temp1.nc mask\_high.nc}

\texttt{cdo lec,1 temp1.nc mask\_low.nc}

\texttt{cdo mul temp1.nc mask\_low.nc temp2.nc}

\texttt{cdo add temp2.nc mask\_high.nc 1-rh\_d\_1-rhc\_bnd.nc }
\newline
Clean some files:
\texttt{rm -f temp?.nc}
}

\frame{
Now we finish the equation by adding the 1-SQRT part:

\texttt{cdo addc,1 -mulc,-1 -sqrt 1-rh\_d\_1-rhc\_bnd.nc
  sundqvistcc.nc}
\newline

check it out:

\texttt{ncview sundqvistcc.nc} 
\newline

Calculate the difference map between Sundqvist CC and Tiedtke CC:

\texttt{cdo sub sundqvistcc.nc cc.nc ccdiff.nc}

\texttt{ncview ccdiff.nc}
}

\section{NCO: more powerful metadata manipulation}

While CDO can change some attributes of variables, it is quite limited
in what it can do regarding the dimensions, variables, metadata, and
global attributes in a netcdf file.  For this task, a much more
powerful set of functions are available collectively in NCO.  NCO
commands also allow you to directly modify the netcdf file, but it is
better to test the function first!

\frame{
As an example of NCO, we presently have a file sundqvist.nc that
contains our diagnostic cloud cover, but the variable name still
refers to relative humidity.  This could be misleading if we pass the
file to someone else, even if the CDO command history is stored in the
global attributes.  

To change the variable name to cloud cover (cc), we can thus use
\texttt{ncrename} (\url{http://linux.die.net/man/1/ncrename}):

\texttt{ncrename -v r,cc -h sundqvistcc.nc}

This is fine but we may also want to change the variable metadate. For
this we use \texttt{ncatted} (see \url{http://linux.die.net/man/1/ncatted}):

\texttt{ncatted -O -a units,cc,o,c,fraction}

\texttt{ncatted -O -a long\_name,cc,o,c,''cloud fraction''}

Note the double quotes if the name has a space.
}




\section{Exercise 2: Some summary statistics}
 \frame{
\alert{Ex 2}:
For the following you will need the new CDO commands 

\texttt{sellonlatbox}
\texttt{fldmean}
\newline

\begin{enumerate}
\item Calculate the mean bias as a function of height (=model level)
\item Calculate the RMS error as a function of height
\item Repeat for the tropics (mod(lat)$<20$ deg) and NH extratropics
  (lat$>30$ deg).
\item Try to make a plot of the PDF of cloud cover - Describe and
  interpret what you find?
\end{enumerate}
}

\section{Exercise 3: Improving the self-consistency}
 \frame{

\alert{Ex 3}:The Tiedtke scheme has liquid and ice cloud water as prognostic
variables.  Basing the CC only on RH will likely lead to non-zero
cloud cover being diagnosed where the cloud water (ice+liquid) is
zero.   In this exercise, use a mask to set the diagnosed Sundqvist
cloud cover to zero in any grid box where the total cloud water
(ice+liquid) is zero.  Repeat the analysis of Exercise 2.  Does the
use of the self-consistent cloud water improve the match of the
Sundqvist and Tiedtke cloud cover?
}

\section{Exercise 4: Impact of subgrid variability}
 \frame{

\alert{Ex 4}:  What parameter in the sundqvist scheme represents the
amount of subgrid-scale variability?  Does subgrid-scale variance of
water increase or decrease when this parameter is increased?
Investigate the impact of setting the threshold for cloud formation to
50, 70, 80 and 90\%, does the cloud fraction respond as you would
expect?  Do you find that one value works best at all heights, or do
you find that different thresholds are appropriate at different
heights?
}


 \frame{
\section{Exercise 5: The Xu and Randall scheme}
\alert{Ex 5}: \cite{xu:96a} used a cloud resolving model (CRM) to derive an
empirical relationship for cloud cover based on the two predictors of
$RH$ and cloud water content:
\begin{equation}
C = RH^p  \left[ 1 - exp \left( \frac{ - \alpha_0 \overline{q_l}}
{(q_s-q_v)^\gamma} \right) \right],
%C = RH^p  \left[ 1 - exp \right]
\end{equation}
where $\gamma$, $\alpha_0$ and $p$ are 'tunable' constants of the
scheme, with values chosen using the CRM data.
\newline

For exercise 3, look up the paper online to get the constants and code
up the scheme.  Repeat the questions in exercise 2 with this new
scheme.   Which scheme is closer to ECMWF's Tiedtke scheme, the Xu or the
Sundqvist scheme?
}

\mode<all>
